% Include packages.
\usepackage[tmargin=2cm,rmargin=1in,lmargin=1in,margin=0.85in,bmargin=2cm,footskip=.2in]{geometry}
\usepackage[english]{babel}                             % Advanced language support
\usepackage{csquotes}                                   % Advanced inline and display quotations (recommended with babel)
\usepackage{isomath}                                    % ISO 80000-2 standardized math notation
\usepackage{mathtools}                                  % Math tools (amsmath extension)
\usepackage{amsthm}                                     % Theorem and proof environments
\usepackage[osf,p,space,sups]{erewhon}                  % Body font settings (erewhon (derived from utopia) for rm)
\usepackage[scaled=0.94,ttdefault,type1]{sourcecodepro} % Body font settings (sourcecodepro for tt)
\usepackage[type1]{sourcesanspro}                       % Body font settings (sourcesanspro for sf)
\usepackage[scaled=1.08,utopia,vvarbb]{newtxmath}       % Math font settings (utopia)
\usepackage{graphicx}                                   % Displaying images and figures
\usepackage[font=sf,justification=centering]{caption}   % Caption typesetting (sans fonts)
\usepackage[font=sf]{floatrow}                          % Floating object environments (sans fonts)
\usepackage{subcaption}                                 % Subcaption typesetting
\usepackage[titles]{tocloft}                            % Advanced control over table of contents
\usepackage{parskip}                                    % Paragraph indentation settings
\usepackage[exponent-product=\cdot]{siunitx}            % SI units (cdot instead of cross for exponent product)
\usepackage{enumitem}                                   % Control over enumerate, itemize, and description environments
\usepackage{listings}                                   % `lstlisting` environment for code
\usepackage{lstautogobble}                              % Autogobble for `listings` package
\usepackage[titletoc]{appendix}                         % Appendices
\usepackage{xcolor}                                     % Various colour settings
\usepackage{anyfontsize}                                % Font size settings (font size substitutions)
\usepackage[most]{tcolorbox}                            % Coloured and framed box environments
\usepackage[showdow,en-US]{datetime2}                   % Date and time settings
%
% Increase linespread.
\linespread{1.04}
%
% Use the `fontenc` package if not using the lualatex compiler.
\ifdefined\directlua\else
    \usepackage[T1]{fontenc}                            % Font encoding
\fi
%
% Bibliography settings (build with lualatex -> biber -> lualatex x 2).
\usepackage[style=authoryear,
    maxbibnames=5,
    maxcitenames=2,
    giveninits=true,
    uniquelist=false,
    uniquename=init,
    seconds=true,
    urldate=iso,
]{biblatex}
\addbibresource{references.bib}
\newcommand\citep{\parencite}
%
% Path to graphics folder.
\graphicspath{{./figures/}}
%
% Fix strange math encoding (lualatex).
\DeclareSIPrefix\micro{\text{\textmu}}{-3}              % Otherwise displays μ as ţ
%
% Display all vectors created using `\vec` with bold italics (given by `\vectorsym`).
\renewcommand{\vec}{\vectorsym}
%
% Commands for gradient, divergence, and curl operators.
\DeclareMathOperator{\Grad}{grad}
\DeclareMathOperator{\Div}{div}
\DeclareMathOperator{\Curl}{curl}
%
% Commands for Big-Oh notation.
\newcommand{\oh}{\mathcal{O}}
%
% Commands for the set of real numbers and Lagrangian/Laplace.
\newcommand{\R}{\mathbb{R}}
\newcommand{\La}{\mathscr{L}}
%
% Shrink bullet points.
\renewcommand\labelitemi{$\vcenter{\hbox{\tiny$\bullet$}}$}
%
% Define TU Delft colors.
\definecolor{tudelft-cyaan}{RGB}{0,166,214}
\definecolor{tudelft-wit}{RGB}{255,255,255}
\definecolor{tudelft-zwart}{RGB}{0,0,0}
\definecolor{tudelft-donkerblauw}{RGB}{12,35,64}
\definecolor{tudelft-turkoois}{RGB}{0,184,200}
\definecolor{tudelft-koningsblauw}{RGB}{0,118,194}
\definecolor{tudelft-paars}{RGB}{111,29,119}
\definecolor{tudelft-roze}{RGB}{239,96,163}
\definecolor{tudelft-bordeaux}{RGB}{165,0,52}
\definecolor{tudelft-rood}{RGB}{224,60,49}
\definecolor{tudelft-oranje}{RGB}{237,104,66}
\definecolor{tudelft-geel}{RGB}{255,184,28}
\definecolor{tudelft-groen}{RGB}{108,194,74}
\definecolor{tudelft-bosgroen}{RGB}{0,155,119}
\definecolor{tudelft-donkergrijs}{RGB}{92,92,92}
%
% Define environment for equation parameters.
\newenvironment{conditions}
{\par\noindent\begin{tabular}{>{$}l<{$} @{${}={}$} l}}
        {\end{tabular}\par\vspace{0.5\belowdisplayskip}}
%
% Define default code environment.
\lstdefinestyle{code}{
    backgroundcolor=\color{white},
    basicstyle=\small\ttfamily,
    breaklines=true,
    captionpos=b,
    commentstyle=\itshape\color{tudelft-donkergrijs},
    escapeinside={\%*}{*)},
    frame=lines,
    keepspaces=true,
    keywordstyle=\color{tudelft-koningsblauw},
    numbers=left,
    numbersep=5pt,
    numberstyle=\small\ttfamily\color{tudelft-donkergrijs},
    rulecolor=\color{black},
    showspaces=false,
    showstringspaces=false,
    stringstyle=\color{tudelft-groen},
    autogobble=true,
    tabsize=2
}
\lstset{style=code}
%
% Define Definition environment. Inspired by SeniorMars, latex_template.
% https://github.com/SeniorMars/dotfiles/tree/main/latex_template.
\makeatletter
\newtcbtheorem[number within=section]{Definition}{Definition}{enhanced,
    before skip=2mm,
    after skip=2mm,
    breakable,
    colback=tudelft-bordeaux!2!white,
    colframe=tudelft-bordeaux,
    boxrule=0.5mm,
    attach boxed title to top left={xshift=1cm,yshift*=1mm-\tcboxedtitleheight},
    fonttitle=\bfseries,
    % varwidth boxed title*=-3cm,
    boxed title style={frame code={
                    \path[fill=tudelft-bordeaux]
                    ([yshift=-1mm,xshift=-1mm]frame.north west)
                    arc[start angle=0,end angle=180,radius=1mm]
                    ([yshift=-1mm,xshift=1mm]frame.north east)
                    arc[start angle=180,end angle=0,radius=1mm];
                    \path[left color=tudelft-bordeaux,right color=tudelft-bordeaux,
                        middle color=tudelft-bordeaux]
                    ([xshift=-2mm]frame.north west) -- ([xshift=2mm]frame.north east)
                    [rounded corners=1mm]-- ([xshift=1mm,yshift=-1mm]frame.north east)
                    -- (frame.south east) -- (frame.south west)
                    -- ([xshift=-1mm,yshift=-1mm]frame.north west)
                    [sharp corners]-- cycle;
                },interior engine=empty,
        },
    #1
}{dfn}
\makeatother
%
% Define Question environment.
\makeatletter
\newtcbtheorem[number within=section]{Question}{Question}{enhanced,
    breakable,
    colback=white,
    colframe=tudelft-koningsblauw,
    attach boxed title to top left={yshift*=-\tcboxedtitleheight},
    fonttitle=\bfseries,
    boxed title size=title,
    boxed title style={%
            sharp corners,
            rounded corners=northwest,
            colback=tcbcolframe,
            boxrule=0pt,
        },
    underlay boxed title={%
            \path[fill=tcbcolframe] (title.south west)--(title.south east)
            to[out=0, in=180] ([xshift=5mm]title.east)--
            (title.center-|frame.east)
            [rounded corners=\kvtcb@arc] |-
            (frame.north) -| cycle;
        },
    #1
}{qs}
\makeatother
%
% Define Claim environment.
\makeatletter
\newtcbtheorem[number within=section]{Claim}{Claim}{enhanced,
    breakable,
    colback = tudelft-groen!2!white,
    frame hidden,
    borderline west = {2pt}{0pt}{tudelft-groen},
    coltitle = tudelft-groen,
    fonttitle = \bfseries\sffamily,
    boxrule = 0sp,
    sharp corners,
    detach title,
    before upper = \tcbtitle\par\smallskip,
    description font = \mdseries,
    separator sign none,
    segmentation style={solid, tudelft-groen},
    #1
}{clm}
%
% Define Example environment.
\makeatletter
\newtcbtheorem[number within=section]{Example}{Example}{enhanced,
    breakable,
    colback = tudelft-bosgroen!2!white,
    colframe = tudelft-bosgroen,
    coltitle = tudelft-bosgroen,
    fonttitle = \bfseries,
    boxrule = 1pt,
    sharp corners,
    detach title,
    before upper = \tcbtitle\par\smallskip,
    description font = \mdseries,
    separator sign none,
    description delimiters parenthesis,
    #1
}{ex}
\makeatother
%
% Define Theorem environment.
\makeatletter
\newtcbtheorem[number within=section]{Theorem}{Theorem}{enhanced,
    breakable,
    colback = tudelft-paars!2!white,
    frame hidden,
    borderline west = {2pt}{0pt}{tudelft-paars},
    coltitle = tudelft-paars,
    fonttitle = \bfseries\sffamily,
    boxrule = 0sp,
    sharp corners,
    detach title,
    before upper = \tcbtitle\par\smallskip,
    description font = \mdseries,
    separator sign none,
    segmentation style={solid, tudelft-paars},
    #1
}{thm}
%
% Define Corollary environment.
\makeatletter
\newtcbtheorem[number within=section]{Corollary}{Corollary}{enhanced,
    breakable,
    colback = tudelft-roze!2!white,
    frame hidden,
    borderline west = {2pt}{0pt}{tudelft-roze},
    coltitle = tudelft-roze,
    fonttitle = \bfseries\sffamily,
    boxrule = 0sp,
    sharp corners,
    detach title,
    before upper = \tcbtitle\par\smallskip,
    description font = \mdseries,
    separator sign none,
    segmentation style={solid, tudelft-roze},
    #1
}{cor}
%
% Define Lemma environment.
\makeatletter
\newtcbtheorem[number within=section]{Lemma}{Lemma}{enhanced,
    breakable,
    colback = tudelft-rood!2!white,
    frame hidden,
    borderline west = {2pt}{0pt}{tudelft-rood},
    coltitle = tudelft-rood,
    fonttitle = \bfseries\sffamily,
    boxrule = 0sp,
    sharp corners,
    detach title,
    before upper = \tcbtitle\par\smallskip,
    description font = \mdseries,
    separator sign none,
    segmentation style={solid, tudelft-rood},
    #1
}{lem}
%
% Define Proposition environment.
\makeatletter
\newtcbtheorem[number within=section]{Proposition}{Proposition}{enhanced,
    breakable,
    colback = tudelft-donkerblauw!2!white,
    frame hidden,
    borderline west = {2pt}{0pt}{tudelft-donkerblauw},
    coltitle = tudelft-donkerblauw,
    fonttitle = \bfseries\sffamily,
    boxrule = 0sp,
    sharp corners,
    detach title,
    before upper = \tcbtitle\par\smallskip,
    description font = \mdseries,
    separator sign none,
    segmentation style={solid, tudelft-donkerblauw},
    #1
}{prp}
%
% Commands for tcolorbox environments.
\newcommand{\dfn}[3]{\begin{Definition}{#1}{#2}#3\end{Definition}}
\newcommand{\qs}[3]{\begin{Question}{#1}{#2}#3\end{Question}}
\newcommand{\sol}{\textbf{\textit{Solution: }}}
\newcommand{\clm}[3]{\begin{Claim}{#1}{#2}#3\end{Claim}}
\newcommand{\ex}[3]{\begin{Example}{#1}{#2}#3\end{Example}}
\newcommand{\thm}[3]{\begin{Theorem}{#1}{#2}#3\end{Theorem}}
\newcommand{\pf}[2]{\begin{proof}[\bfseries Proof #1:]#2\end{proof}}
\newcommand{\cor}[3]{\begin{Corollary}{#1}{#2}#3\end{Corollary}}
\newcommand{\lem}[3]{\begin{Lemma}{#1}{#2}#3\end{Lemma}}
\newcommand{\prp}[3]{\begin{Proposition}{#1}{#2}#3\end{Proposition}}
%
% Hyperref setup (swap comments for coloured links).
\usepackage[hidelinks]{hyperref}
% \usepackage[colorlinks,allcolors=tudelft-cyaan]{hyperref}
%
% Commands for differentials. Redefines the underdot command! Should be below `hyperref`.
\renewcommand\d{\mathop{}\!\mathrm{d}}
\newcommand\p{\mathop{}\!\mathrm{\partial}}
%