% Include packages.
\usepackage[tmargin=2cm,rmargin=1in,lmargin=1in,margin=0.85in,bmargin=2cm,footskip=.2in]{geometry}
\usepackage{isomath}                                % ISO 80000-2 standardized math notation
\usepackage{mathtools}                              % Math tools (amsmath extension)
\usepackage{amsthm}                                 % Theorem and proof environments
\usepackage{sourcesanspro,sourcecodepro,XCharter}   % Body font settings (sourcesanspro for sf, sourcecodepro for tt, XCharter for rm)
\usepackage[xcharter,vvarbb]{newtxmath}             % Math font settings (XCharter)
\usepackage{graphicx}                               % Displaying images and figures
\usepackage[font=sf]{caption}                       % Caption typesetting (sans fonts)
\usepackage[font=sf]{floatrow}                      % Floating object environments (sans fonts)
\usepackage{subcaption}                             % Subcaption typesetting
\usepackage{parskip}                                % Paragraph indentation settings
\usepackage[exponent-product=\cdot]{siunitx}        % SI units (cdot instead of cross for exponent product)
\usepackage{enumitem}                               % Control over enumerate, itemize, and description environments
\usepackage{listings}                               % `lstlisting` environment for code
\usepackage{lstautogobble}                          % Autogobble for `listings` package
\usepackage[titletoc]{appendix}                     % Appendices
\usepackage{xcolor}                                 % Various colour settings
\usepackage{anyfontsize}                            % Font size settings (font size substitutions)
\usepackage[most]{tcolorbox}                        % Coloured and framed box environments
%
% Use the `fontenc` package if not using the lualatex compiler.
\ifdefined\directlua\else
  \usepackage[T1]{fontenc}                          % Font encoding
\fi
%
% Bibliography settings (build with lualatex -> biber -> lualatex x 2).
\usepackage[style=authoryear,
    maxbibnames=5,
    maxcitenames=2,
    giveninits=true,
    uniquelist=false,
    uniquename=init,
    seconds=true,
    urldate=iso,
    ]{biblatex}
\addbibresource{references.bib}
\newcommand\citep{\parencite}
%
% Hyperref setup.
\usepackage[hidelinks]{hyperref}
\hypersetup{pdfauthor={Author Name},pdftitle={Title}}
%
% Path to graphics folder.
\graphicspath{{./figures/}}
%
% Commands for differentials. Redefines the underdot command!
\renewcommand\d{\mathop{}\!\mathrm{d}}
\newcommand\p{\mathop{}\!\mathrm{\partial}}
%
% Fix strange math encoding (lualatex).
\DeclareSIPrefix\micro{\text{\textmu}}{-3}          % Otherwise displays μ as ţ
%
% Commands for Big-Oh notation.
\newcommand{\oh}{\mathcal{O}}
%
% Commands for the set of real numbers and Lagrangian/Laplace.
\newcommand{\R}{\mathbb{R}}
\newcommand{\La}{\mathscr{L}}
%
% Shrink bullet points.
\renewcommand\labelitemi{$\vcenter{\hbox{\tiny$\bullet$}}$}
%
% Define TU Delft colors.
\definecolor{tudelft-cyaan}{RGB}{0,166,214}
\definecolor{tudelft-wit}{RGB}{255,255,255}
\definecolor{tudelft-zwart}{RGB}{0,0,0}
\definecolor{tudelft-donkerblauw}{RGB}{12,35,64}
\definecolor{tudelft-turkoois}{RGB}{0,184,200}
\definecolor{tudelft-koningsblauw}{RGB}{0,118,194}
\definecolor{tudelft-paars}{RGB}{111,29,119}
\definecolor{tudelft-roze}{RGB}{239,96,163}
\definecolor{tudelft-bordeaux}{RGB}{165,0,52}
\definecolor{tudelft-rood}{RGB}{224,60,49}
\definecolor{tudelft-oranje}{RGB}{237,104,66}
\definecolor{tudelft-geel}{RGB}{255,184,28}
\definecolor{tudelft-groen}{RGB}{108,194,74}
\definecolor{tudelft-bosgroen}{RGB}{0,155,119}
\definecolor{tudelft-donkergrijs}{RGB}{92,92,92}
%
% Define default code environment.
\lstdefinestyle{code}{
    backgroundcolor=\color{white},
    basicstyle=\small\ttfamily,
    breaklines=true,
    captionpos=b,
    commentstyle=\itshape\color{tudelft-donkergrijs},
    escapeinside={\%*}{*)},
    frame=lines,
    keepspaces=true,
    keywordstyle=\color{tudelft-koningsblauw},
    numbers=left,
    numbersep=5pt,
    numberstyle=\small\ttfamily\color{tudelft-donkergrijs},
    rulecolor=\color{black},
    showspaces=false,
    showstringspaces=false,
    stringstyle=\color{tudelft-groen},
    autogobble=true,
    tabsize=2
}
\lstset{style=code}
%
% Define Question environment. Inspired by `SeniorMars`.
\makeatletter
\newtcbtheorem[number within=section]{Question}{Question}{enhanced,
    label=qs:\thesection.\arabic{\tcbcounter},
    breakable,
    colback=white,
    colframe=tudelft-koningsblauw,
    attach boxed title to top left={yshift*=-\tcboxedtitleheight},
    fonttitle=\bfseries,
    boxed title size=title,
    boxed title style={%
            sharp corners,
            rounded corners=northwest,
            colback=tcbcolframe,
            boxrule=0pt,
        },
    underlay boxed title={%
            \path[fill=tcbcolframe] (title.south west)--(title.south east)
            to[out=0, in=180] ([xshift=5mm]title.east)--
            (title.center-|frame.east)
            [rounded corners=\kvtcb@arc] |-
            (frame.north) -| cycle;
        },
    #1
}{def}
\makeatother
%
% Define Definition environment.
\makeatletter
\newtcbtheorem[number within=section]{Definition}{Definition}{enhanced,
    label=dfn:\thesection.\arabic{\tcbcounter},
    before skip=2mm,
    after skip=2mm,
    breakable,
    colback=tudelft-bordeaux!2!white,
    colframe=tudelft-bordeaux,
    boxrule=0.5mm,
	attach boxed title to top left={xshift=1cm,yshift*=1mm-\tcboxedtitleheight},
    fonttitle=\bfseries,
    % varwidth boxed title*=-3cm,
    boxed title style={frame code={
					\path[fill=tudelft-bordeaux]
					([yshift=-1mm,xshift=-1mm]frame.north west)
					arc[start angle=0,end angle=180,radius=1mm]
					([yshift=-1mm,xshift=1mm]frame.north east)
					arc[start angle=180,end angle=0,radius=1mm];
					\path[left color=tudelft-bordeaux,right color=tudelft-bordeaux,
						middle color=tudelft-bordeaux]
					([xshift=-2mm]frame.north west) -- ([xshift=2mm]frame.north east)
					[rounded corners=1mm]-- ([xshift=1mm,yshift=-1mm]frame.north east)
					-- (frame.south east) -- (frame.south west)
					-- ([xshift=-1mm,yshift=-1mm]frame.north west)
					[sharp corners]-- cycle;
				},interior engine=empty,
		},
    #1
}{def}
\makeatother
%
% Define Example environment.
\makeatletter
\newtcbtheorem[number within=section]{Example}{Example}{enhanced,
    label=ex:\thesection.\arabic{\tcbcounter},
    breakable,
    colback = tudelft-bosgroen!2!white,
    colframe = tudelft-bosgroen,
    coltitle = tudelft-bosgroen,
    fonttitle = \bfseries,
    boxrule = 1pt,
    sharp corners,
    detach title,
    before upper = \tcbtitle\par\smallskip,
    description font = \mdseries,
    separator sign none,
    description delimiters parenthesis,
    #1
}{def}
\makeatother
%
% Define Theorem environment.
\makeatletter
\newtcbtheorem[number within=section]{Theorem}{Theorem}{enhanced,
    label=thm:\thesection.\arabic{\tcbcounter},
    breakable,
    colback = tudelft-paars!2!white,
    frame hidden,
    borderline west = {2pt}{0pt}{tudelft-paars},
    coltitle = tudelft-paars,
    fonttitle = \bfseries\sffamily,
    boxrule = 0sp,
    sharp corners,
    detach title,
    before upper = \tcbtitle\par\smallskip,
    description font = \mdseries,
    separator sign none,
    segmentation style={solid, tudelft-paars},
    #1
}{def}
\makeatother
%
% Commands for tcolorbox environments.
\newcommand{\qs}[2]{\begin{Question}{#1}{}#2\end{Question}}
\newcommand{\dfn}[2]{\begin{Definition}{#1}{}#2\end{Definition}}
\newcommand{\ex}[2]{\begin{Example}{#1}{}#2\end{Example}}
\newcommand{\thm}[2]{\begin{Theorem}{#1}{}#2\end{Theorem}}
\newcommand{\pf}[2]{\begin{proof}[\bfseries Proof #1:]#2\end{proof}}
\newcommand{\sol}{\textbf{\textit{Solution: }}}