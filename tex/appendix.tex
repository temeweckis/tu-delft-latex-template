\begin{appendices}

    \section{Example Python script}

    \begin{lstlisting}[language=Python, caption={Example Python script \citep{tudelftopencourseware}}]
        # Program      : Euler's method
        # Author       : MOOC team Mathematical Modelling Basics
        # Created      : April, 2017
        
        import numpy as np
        import matplotlib.pyplot as plt
        
        print("Solution for dP/dt = 0.7*P")	# in Python 2.7: use no brackets
        
        # Initializations
        
        Dt = 0.1                                # timestep Delta t
        P_init = 10                             # initial population 
        t_init = 0                              # initial time
        t_end = 5                               # stopping time
        n_steps = int(round((t_end-t_init)/Dt)) # total number of timesteps
        
        t_arr = np.zeros(n_steps + 1)           # create an array of zeros for t
        P_arr = np.zeros(n_steps + 1)           # create an array of zeros for P
        t_arr[0] = t_init                       # add the initial P to the array
        P_arr[0] = P_init                       # add the initial t to the array
        
        # Euler's method
        
        for i in range (1, n_steps + 1):
            P = P_arr[i-1]
            t = t_arr[i-1]
            dPdt = 0.7*P                        # calculate the derivative 
            P_arr[i] = P + Dt*dPdt              # calculate P on the next time step
            t_arr[i] = t + Dt                   # adding the new t-value to the list
        
        # Plot the results
        
        fig = plt.figure()                      # create figure
        plt.plot(t_arr, P_arr, linewidth = 4)   # plot population vs. time
        
        plt.title('dP/dt = 0.7P, P(0)=10', fontsize = 25)  
        plt.xlabel('t (in days)', fontsize = 20)
        plt.ylabel('P(t)', fontsize = 20)
        
        plt.xticks(fontsize = 15)
        plt.yticks(fontsize = 15)
        plt.grid(True)                          # show grid 
        plt.axis([0, 5, 0, 200])                # define the axes
        plt.show()                              # show the plot
        # save the figure as .jpg
        fig.savefig('Rainbowfish.jpg', dpi=fig.dpi, bbox_inches = "tight")
    \end{lstlisting}
    
\end{appendices}